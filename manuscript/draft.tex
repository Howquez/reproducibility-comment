% Options for packages loaded elsewhere
\PassOptionsToPackage{unicode}{hyperref}
\PassOptionsToPackage{hyphens}{url}
\PassOptionsToPackage{dvipsnames,svgnames,x11names}{xcolor}
%
\documentclass[
  twocolumn]{article}

\usepackage{amsmath,amssymb}
\usepackage{iftex}
\ifPDFTeX
  \usepackage[T1]{fontenc}
  \usepackage[utf8]{inputenc}
  \usepackage{textcomp} % provide euro and other symbols
\else % if luatex or xetex
  \usepackage{unicode-math}
  \defaultfontfeatures{Scale=MatchLowercase}
  \defaultfontfeatures[\rmfamily]{Ligatures=TeX,Scale=1}
\fi
\usepackage{lmodern}
\ifPDFTeX\else  
    % xetex/luatex font selection
\fi
% Use upquote if available, for straight quotes in verbatim environments
\IfFileExists{upquote.sty}{\usepackage{upquote}}{}
\IfFileExists{microtype.sty}{% use microtype if available
  \usepackage[]{microtype}
  \UseMicrotypeSet[protrusion]{basicmath} % disable protrusion for tt fonts
}{}
\makeatletter
\@ifundefined{KOMAClassName}{% if non-KOMA class
  \IfFileExists{parskip.sty}{%
    \usepackage{parskip}
  }{% else
    \setlength{\parindent}{0pt}
    \setlength{\parskip}{6pt plus 2pt minus 1pt}}
}{% if KOMA class
  \KOMAoptions{parskip=half}}
\makeatother
\usepackage{xcolor}
\setlength{\emergencystretch}{3em} % prevent overfull lines
\setcounter{secnumdepth}{-\maxdimen} % remove section numbering
% Make \paragraph and \subparagraph free-standing
\makeatletter
\ifx\paragraph\undefined\else
  \let\oldparagraph\paragraph
  \renewcommand{\paragraph}{
    \@ifstar
      \xxxParagraphStar
      \xxxParagraphNoStar
  }
  \newcommand{\xxxParagraphStar}[1]{\oldparagraph*{#1}\mbox{}}
  \newcommand{\xxxParagraphNoStar}[1]{\oldparagraph{#1}\mbox{}}
\fi
\ifx\subparagraph\undefined\else
  \let\oldsubparagraph\subparagraph
  \renewcommand{\subparagraph}{
    \@ifstar
      \xxxSubParagraphStar
      \xxxSubParagraphNoStar
  }
  \newcommand{\xxxSubParagraphStar}[1]{\oldsubparagraph*{#1}\mbox{}}
  \newcommand{\xxxSubParagraphNoStar}[1]{\oldsubparagraph{#1}\mbox{}}
\fi
\makeatother


\providecommand{\tightlist}{%
  \setlength{\itemsep}{0pt}\setlength{\parskip}{0pt}}\usepackage{longtable,booktabs,array}
\usepackage{calc} % for calculating minipage widths
% Correct order of tables after \paragraph or \subparagraph
\usepackage{etoolbox}
\makeatletter
\patchcmd\longtable{\par}{\if@noskipsec\mbox{}\fi\par}{}{}
\makeatother
% Allow footnotes in longtable head/foot
\IfFileExists{footnotehyper.sty}{\usepackage{footnotehyper}}{\usepackage{footnote}}
\makesavenoteenv{longtable}
\usepackage{graphicx}
\makeatletter
\newsavebox\pandoc@box
\newcommand*\pandocbounded[1]{% scales image to fit in text height/width
  \sbox\pandoc@box{#1}%
  \Gscale@div\@tempa{\textheight}{\dimexpr\ht\pandoc@box+\dp\pandoc@box\relax}%
  \Gscale@div\@tempb{\linewidth}{\wd\pandoc@box}%
  \ifdim\@tempb\p@<\@tempa\p@\let\@tempa\@tempb\fi% select the smaller of both
  \ifdim\@tempa\p@<\p@\scalebox{\@tempa}{\usebox\pandoc@box}%
  \else\usebox{\pandoc@box}%
  \fi%
}
% Set default figure placement to htbp
\def\fps@figure{htbp}
\makeatother
% definitions for citeproc citations
\NewDocumentCommand\citeproctext{}{}
\NewDocumentCommand\citeproc{mm}{%
  \begingroup\def\citeproctext{#2}\cite{#1}\endgroup}
\makeatletter
 % allow citations to break across lines
 \let\@cite@ofmt\@firstofone
 % avoid brackets around text for \cite:
 \def\@biblabel#1{}
 \def\@cite#1#2{{#1\if@tempswa , #2\fi}}
\makeatother
\newlength{\cslhangindent}
\setlength{\cslhangindent}{1.5em}
\newlength{\csllabelwidth}
\setlength{\csllabelwidth}{3em}
\newenvironment{CSLReferences}[2] % #1 hanging-indent, #2 entry-spacing
 {\begin{list}{}{%
  \setlength{\itemindent}{0pt}
  \setlength{\leftmargin}{0pt}
  \setlength{\parsep}{0pt}
  % turn on hanging indent if param 1 is 1
  \ifodd #1
   \setlength{\leftmargin}{\cslhangindent}
   \setlength{\itemindent}{-1\cslhangindent}
  \fi
  % set entry spacing
  \setlength{\itemsep}{#2\baselineskip}}}
 {\end{list}}
\usepackage{calc}
\newcommand{\CSLBlock}[1]{\hfill\break\parbox[t]{\linewidth}{\strut\ignorespaces#1\strut}}
\newcommand{\CSLLeftMargin}[1]{\parbox[t]{\csllabelwidth}{\strut#1\strut}}
\newcommand{\CSLRightInline}[1]{\parbox[t]{\linewidth - \csllabelwidth}{\strut#1\strut}}
\newcommand{\CSLIndent}[1]{\hspace{\cslhangindent}#1}

\usepackage{titling}
\usepackage{sectsty}
\usepackage{lettrine}
\usepackage{flushend}
\usepackage[all]{nowidow}
\usepackage{abstract}
\renewcommand{\abstractname}{}
\renewcommand{\absnamepos}{empty}
\renewcommand{\LettrineTextFont}{\normalfont}
\allsectionsfont{\rmfamily}
\renewcommand{\familydefault}{\rmdefault}
\pretitle{\begin{center}\LARGE\rmfamily}
\posttitle{\end{center}}
\predate{\begin{center}\rmfamily}
\postdate{\end{center}}
\setlength{\parindent}{0.25in}
\setlength{\parskip}{0pt}
\renewcommand{\thefootnote}{}
\pretitle{\begin{center}\LARGE\rmfamily \vspace{1em}}
\posttitle{\vspace{1em} \end{center}}
\predate{\begin{center}\rmfamily \vspace{1em}}
\postdate{\vspace{2em} \end{center}}
\makeatletter
\@ifpackageloaded{caption}{}{\usepackage{caption}}
\AtBeginDocument{%
\ifdefined\contentsname
  \renewcommand*\contentsname{Table of contents}
\else
  \newcommand\contentsname{Table of contents}
\fi
\ifdefined\listfigurename
  \renewcommand*\listfigurename{List of Figures}
\else
  \newcommand\listfigurename{List of Figures}
\fi
\ifdefined\listtablename
  \renewcommand*\listtablename{List of Tables}
\else
  \newcommand\listtablename{List of Tables}
\fi
\ifdefined\figurename
  \renewcommand*\figurename{Figure}
\else
  \newcommand\figurename{Figure}
\fi
\ifdefined\tablename
  \renewcommand*\tablename{Table}
\else
  \newcommand\tablename{Table}
\fi
}
\@ifpackageloaded{float}{}{\usepackage{float}}
\floatstyle{ruled}
\@ifundefined{c@chapter}{\newfloat{codelisting}{h}{lop}}{\newfloat{codelisting}{h}{lop}[chapter]}
\floatname{codelisting}{Listing}
\newcommand*\listoflistings{\listof{codelisting}{List of Listings}}
\makeatother
\makeatletter
\makeatother
\makeatletter
\@ifpackageloaded{caption}{}{\usepackage{caption}}
\@ifpackageloaded{subcaption}{}{\usepackage{subcaption}}
\makeatother

\usepackage{bookmark}

\IfFileExists{xurl.sty}{\usepackage{xurl}}{} % add URL line breaks if available
\urlstyle{same} % disable monospaced font for URLs
\hypersetup{
  pdftitle={Prioritizing computational reproducibility in behavioral science},
  pdfauthor={Hauke Roggenkamp ; Susanne Adler ; Michael V. Reiss ; Stefan Feuerriegel ; Nicolas Pröllochs ; Claire Robertson ; Felix Holzmeister },
  colorlinks=true,
  linkcolor={blue},
  filecolor={Maroon},
  citecolor={Blue},
  urlcolor={Blue},
  pdfcreator={LaTeX via pandoc}}


\title{Prioritizing computational reproducibility in behavioral science}
\author{Hauke Roggenkamp
\href{https://orcid.org/0009-0005-5176-4718}{\includegraphics[width=0.018\linewidth,height=\textheight,keepaspectratio]{../misc/orcid.png}} \and Susanne
Adler
\href{https://orcid.org/0000-0002-3211-6871}{\includegraphics[width=0.018\linewidth,height=\textheight,keepaspectratio]{../misc/orcid.png}} \and Michael
V. Reiss
\href{https://orcid.org/0000-0002-6094-9985}{\includegraphics[width=0.018\linewidth,height=\textheight,keepaspectratio]{../misc/orcid.png}} \and Stefan
Feuerriegel
\href{https://orcid.org/0000-0001-7856-8729}{\includegraphics[width=0.018\linewidth,height=\textheight,keepaspectratio]{../misc/orcid.png}} \and Nicolas
Pröllochs
\href{https://orcid.org/0000-0002-1835-7302}{\includegraphics[width=0.018\linewidth,height=\textheight,keepaspectratio]{../misc/orcid.png}} \and Claire
Robertson
\href{https://orcid.org/0000-0001-8403-6358}{\includegraphics[width=0.018\linewidth,height=\textheight,keepaspectratio]{../misc/orcid.png}} \and Felix
Holzmeister
\href{https://orcid.org/0000-0001-9606-0427}{\includegraphics[width=0.018\linewidth,height=\textheight,keepaspectratio]{../misc/orcid.png}}}
\date{July 3, 2025}

\begin{document}

\makeatletter
\twocolumn[
\maketitle
\begin{@twocolumnfalse}

\begin{abstract}
% Write your abstract here ----------------------------------------------------

\vspace{1em}

\normalsize \noindent Standfirst (in lieu of an abstract) goes here (up to 360 characters, including spaces). %While behavioral and social sciences have made progress addressing credibility challenges, computational reproducibility—the ability to verify results using original data and code—remains inadequately addressed. In this comment, we draw on our experience as both reproducers and authors to propose a "computational empathy" mindset with four key practices: comprehensive documentation, literate programming, structured organization, and environment management. These practices require minimal technical knowledge yet considerably improve reproducibility, allowing researchers to make their contributions more credible, transparent, and cumulative.



% abstract completed  ----------------------------------------------------------
\end{abstract}

\vspace{3em}


\end{@twocolumnfalse}
]
\makeatother


\footnotetext{Hauke Roggenkamp, PhD Candidate in Marketing at the University of St. Gallen, St. Gallen, Switzerland (email: \href{mailto:Hauke.Roggenkamp@unisg.ch}{Hauke.Roggenkamp@unisg.ch}). Michael V. Reiss, Postdoctoral Researcher in Communication Science at the Leibniz Institute for Media Research, Hamburg, Germany. Susanne Adler, Postdoctoral Researcher in Marketing at the LMU Munich, Munich, Germany. Stefan Feuerriegel, Full Professor at the School of Management and the Faculty of Mathematics, Informatics, and Statistics at LMU Munich, Munich, Germany. Nicolas Pröllochs, Professor of Data Science and Digitization at the University of Gießen, Gießen, Germany. Claire Robertson, Postdoctoral Researcher in Social Psychology at NYU, New York, USA. Felix Holzmeister, Assistant Professor of Behavioral and Experimental Economics and Finance at the University of Innsbruck, Innsbruck, Austria.}

\lettrine[lines=3]{T}{ he} behavioral sciences have demonstrated
remarkable progress in response to credibility challenges over the past
decade (Open Science Collaboration 2015; Camerer et al. 2016, 2018;
Klein et al. 2018). Specifically, the open science movement,
pre-registration, and crowd-sourced science have transformed research
practices and strengthened methodological rigor. One of these
crowd-sourced collaborations is featured in this issue and reports that
x percent of published findings replicate/are reproducible etc. (see the
I4Rs meta-paper).

Whereas replicability describes the ability to obtain consistent results
using new data, \emph{reproducibility} is a more fundamental criterion
that comes in two complementary forms (Dreber and Johannesson 2025).
Computational reproducibility describes the extent to which one can
recreate numerically identical results using the same data and code as
the original authors. As such, it verifies the mapping between data,
code, and reported results. Recreate reproducibility ignores the
original authors' code and describes the extent to which one can
recreate numerically identical results using the same data and only the
published methodological descriptions. As such, it verifies the mapping
between reported methods and reported results, catching gaps in
methodological reporting or coding errors that computational
reproducibility may miss.

More fundamentally, the critical standard is whether not whether the
results but the \emph{conclusions and claims} remain consistent. Rather
than adopting a dichotomous pass/fail approach (analogous to the widely
applied binary treatment of statistical significance) we advocate for
viewing reproducibility as existing on a spectrum. This includes
assessing the degree of preservation across: (1) the statistical
significance and direction of primary results, (2) the substantive
magnitude of estimated effects, and (3) the overall interpretation and
policy implications.\footnote{When numerical discrepancies do arise, we
  recommend applying a principle of in \emph{dubio pro reo}: giving
  authors the benefit of the doubt and engage in respectful dialogue
  with original authors to clarify methodological details before
  concluding that reproduction has failed.}

As behavioral science continues addressing broader credibility
challenges, ensuring a mapping of methods, code and results represents a
fundamental step toward transparent, integrative research. Accordingly,
these two types reproducibility should be a minimal requirement during
the peer review process, which must evaluate not only the
appropriateness of the methods, but also verify that the code implements
these methods accurately and that they produce the reported results.
Yet, when put to the test, more than half of 2,091 replication packages
retrieved from the Harvard Dataverse repository fail to execute
(Trisovic et al. 2022). {[}Consider other sources of \emph{real}
reproducibility issues here.{]}

Why, despite its importance, do so many projects fail to achieve
reproducibility? The challenge likely stems from both technical barriers
and researcher priorities. On the technical side, rapidly evolving
software environments, complex dependency chains, and inconsistent
standards create indeed some obstacles (Epskamp 2019). However, these
technical challenges are compounded when researchers hesitate to invest
in reproducible workflows due to perceived time constraints, assumptions
about limited reuse value, or concerns about required technical
expertise. These reservations, while understandable, reflect
misperceptions about both the effort required and the benefits gained.
While the time investment is frontloaded, adopting a reproducible
workflow right from the outset of a project can actually increase
efficiency, reduce errors, and, ultimately, save time. Moreover, the
primary beneficiary of reproducible code is often the researcher
themselves, when revisiting analyses months or even years later. For
this reason, we suggest a mindset Vilhuber (2021) coins ``computational
empathy''. This concept encourages researchers to structure their work
as if handing it to someone with their background but no
project-specific knowledge, or more pragmatically, to their future self
who will inevitably forget project details during the review process.

Building on this idea, we propose a phase-based framework that
translates computational empathy into concrete actions to increase
reproducibility. Our approach distinguishes between minimal standards,
that should be achievable by all researchers regardless of technical
background, and best practices for those with more advanced skills. This
structure is intended to acknowledge varying technical skills (or time
constraints) while ensuring basic reproducibility remains accessible.

\begin{longtable}[]{@{}
  >{\raggedright\arraybackslash}p{(\linewidth - 4\tabcolsep) * \real{0.3333}}
  >{\raggedright\arraybackslash}p{(\linewidth - 4\tabcolsep) * \real{0.1667}}
  >{\raggedright\arraybackslash}p{(\linewidth - 4\tabcolsep) * \real{0.5000}}@{}}
\toprule\noalign{}
\begin{minipage}[b]{\linewidth}\raggedright
Requirement
\end{minipage} & \begin{minipage}[b]{\linewidth}\raggedright
When
\end{minipage} & \begin{minipage}[b]{\linewidth}\raggedright
Specific Actions
\end{minipage} \\
\midrule\noalign{}
\endhead
\bottomrule\noalign{}
\endlastfoot
AVAILABILITY & Before starting & • Create project repository with
standard structure (code/, data/, output/, docs/) \\
& • Set up version control (git init) and link to GitHub/GitLab & \\
& • Choose data repository (Zenodo/OSF/Dataverse) and reserve DOI & \\
While working & • Commit code changes daily with descriptive messages
& \\
& • Use git-lfs or external storage for large data files
(\textgreater100MB) & \\
& • Save raw data separately from processed data & \\
Before submission & • Upload complete dataset with documentation to
repository & \\
& • Ensure all code files are tracked in version control & \\
& • Test that repository contains everything needed to reproduce & \\
\end{longtable}

TRACEABILITY \textbar{} Before starting \textbar{} • Create README
template with sections for: requirements, file structure, execution
order \textbar{} \textbar{} • Decide on naming conventions for files and
variables \textbar{} While working \textbar{} • Number scripts in
execution order (01\_clean\_data.R, 02\_analysis.R) \textbar{}
\textbar{} • Document each analysis decision as you make it \textbar{}
\textbar{} • Create a master script that runs entire pipeline \textbar{}
Before submission \textbar{} • Write step-by-step execution instructions
in README \textbar{} \textbar{} • Create flowchart showing data/code
dependencies \textbar{} \textbar{} • Include expected runtime and output
descriptions

EXPLAINABILITY \textbar{} Before starting \textbar{} • Choose literate
programming tool (Jupyter, R Markdown, Quarto) \textbar{} \textbar{} •
Set up computational notebook for main analyses \textbar{} While working
\textbar{} • Write explanatory text before each code chunk \textbar{}
\textbar{} • Avoid all manual data manipulation (no Excel edits)
\textbar{} \textbar{} • Include sanity checks and assertions in code
\textbar{} Before submission \textbar{} • Add comments explaining
non-obvious code logic \textbar{} \textbar{} • Document any deviations
from pre-registration \textbar{} \textbar{} • Include interpretation of
key results in notebook

CONTAINERIZATION \textbar{} Before starting \textbar{} • Create
requirements.txt/renv.lock file \textbar{} \textbar{} • Document
operating system and software versions \textbar{} \textbar{} • Use
relative paths exclusively (./data not C:/Users/\ldots) \textbar{} While
working \textbar{} • Update dependency list when adding packages
\textbar{} \textbar{} • Test code regularly in fresh environment
\textbar{} \textbar{} • Avoid system-specific commands or paths
\textbar{} Before submission \textbar{} • Create Docker container or
virtual environment \textbar{} \textbar{} • Test entire pipeline on
different operating system \textbar{} \textbar{} • Include installation
instructions for all dependencies

\section{References}\label{references}

\phantomsection\label{refs}
\begin{CSLReferences}{1}{0}
\bibitem[\citeproctext]{ref-AuerEtAl_2024}
Auer, Tobias, Maria Ulasik, and Felix Holzmeister. 2024. {``A Comment on
"Motivated Errors" by Exley and Kessler (2024).''} I4R Discussion Paper
Series 161. The Institute for Replication (I4R).
\url{https://ideas.repec.org/p/zbw/i4rdps/161.html}.

\bibitem[\citeproctext]{ref-BrodeurEtAl_2024}
Brodeur, Abel, Anna Dreber, Fernando Hoces de la Guardia, and Edward
Miguel. 2024. {``Reproduction and Replication at Scale.''}
Correspondence. \emph{Nature Human Behaviour} 8 (January): 2--3.
\url{https://doi.org/10.1038/s41562-023-01807-2}.

\bibitem[\citeproctext]{ref-CamererEtAl_2016}
Camerer, Colin F., Anna Dreber, Eskil Forsell, Teck-Hua Ho, Jürgen
Huber, Magnus Johannesson, Michael Kirchler, et al. 2016. {``Evaluating
Replicability of Laboratory Experiments in Economics.''} \emph{Science}
351 (6280): 1433--36. \url{https://doi.org/10.1126/science.aaf0918}.

\bibitem[\citeproctext]{ref-CamererEtAl_2018}
Camerer, Colin F., Anna Dreber, Felix Holzmeister, Teck-Hua Ho, Jürgen
Huber, Magnus Johannesson, Michael Kirchler, et al. 2018. {``Evaluating
the Replicability of Social Science Experiments in {Nature} and
{Science} Between 2010 and 2015.''} Letter. \emph{Nature Human
Behaviour} 2 (August): 637--44.
\url{https://doi.org/10.1038/s41562-018-0399-z}.

\bibitem[\citeproctext]{ref-DeerAdlerDattaEtAl_2025}
Deer, Lachlan, Susanne J. Adler, Hannes Datta, Natalie Mizik, and Marko
Sarstedt. 2025. {``Toward Open Science in Marketing Research.''}
\emph{International Journal of Research in Marketing} 42 (1): 212--33.
\url{https://doi.org/10.1016/j.ijresmar.2024.12.005}.

\bibitem[\citeproctext]{ref-DreberJohannesson_2024}
Dreber, Anna, and Magnus Johannesson. 2025. {``A Framework for
Evaluating Reproducibility and Replicability in Economics.''}
\emph{Economic Inquiry} 63 (2): 338--56.
https://doi.org/\url{https://doi.org/10.1111/ecin.13244}.

\bibitem[\citeproctext]{ref-Epskamp_2019}
Epskamp, Sacha. 2019. {``Reproducibility and Replicability in a
Fast-Paced Methodological World.''} \emph{Advances in Methods and
Practices in Psychological Science} 2 (2): 145--55.
\url{https://doi.org/10.1177/2515245919847421}.

\bibitem[\citeproctext]{ref-KleinEtAl_2018}
Klein, Richard A., Michelangelo Vianello, Fred Hasselman, Byron G.
Adams, Reginald B. AdamsJr., Sinan Alper, Mark Aveyard, et al. 2018.
{``Many Labs 2: Investigating Variation in Replicability Across Samples
and Settings.''} \emph{Advances in Methods and Practices in
Psychological Science} 1 (4): 443--90.
\url{https://doi.org/10.1177/2515245918810225}.

\bibitem[\citeproctext]{ref-Knuth_1984}
Knuth, D. E. 1984. {``Literate Programming.''} \emph{The Computer
Journal} 27 (2): 97--111. \url{https://doi.org/10.1093/comjnl/27.2.97}.

\bibitem[\citeproctext]{ref-LeBelEtAl_2018}
LeBel, Etienne P., Randy J. McCarthy, Brian D. Earp, Malte Elson, and
Wolf Vanpaemel. 2018. {``A Unified Framework to Quantify the Credibility
of Scientific Findings.''} \emph{Advances in Methods and Practices in
Psychological Science} 1 (3): 389--402.
\url{https://doi.org/10.1177/2515245918787489}.

\bibitem[\citeproctext]{ref-OpenScienceCollaboration_2015}
Open Science Collaboration. 2015. {``Estimating the Reproducibility of
Psychological Science.''} \emph{Science} 349 (6251): aac4716.
\url{https://doi.org/10.1126/science.aac4716}.

\bibitem[\citeproctext]{ref-ParsonsEtAl_2022}
Parsons, Sam, Flávio Azevedo, Mahmoud M. Elsherif, Samuel Guay, Owen N.
Shahim, Gisela H. Govaart, Emma Norris, et al. 2022. {``A
Community-Sourced Glossary of Open Scholarship Terms.''} Comment.
\emph{Nature Human Behaviour} 6: 312--18.
\url{https://doi.org/10.1038/s41562-021-01269-4}.

\bibitem[\citeproctext]{ref-binder}
Project Jupyter, Matthias Bussonnier, Jessica Forde, Jeremy Freeman,
Brian Granger, Tim Head, Chris Holdgraf, et al. 2018. {``{B}inder 2.0 -
{R}eproducible, Interactive, Sharable Environments for Science at
Scale.''} In \emph{{P}roceedings of the 17th {P}ython in {S}cience
{C}onference}, edited by Fatih Akici, David Lippa, Dillon Niederhut, and
M Pacer, 113--20.
\href{https://doi.org/\%2010.25080/Majora-4af1f417-011\%20}{https://doi.org/
10.25080/Majora-4af1f417-011 }.

\bibitem[\citeproctext]{ref-promoting2024}
{``Promoting Reproduction and Replication at Scale.''} 2024.
\emph{Nature Human Behaviour} 8 (1).
\url{https://doi.org/10.1038/s41562-024-01818-7}.

\bibitem[\citeproctext]{ref-ReissRoggenkamp_2025}
Reiss, Michael V., and Hauke Roggenkamp. 2025. {``A Comment on
"Negativity Drives Online News Consumption".''} I4R Discussion Paper
Series 199. Institute for Replication (I4R).
\url{https://hdl.handle.net/10419/311304}.

\bibitem[\citeproctext]{ref-RobertsonEtAl_2023}
Robertson, Claire E., Nicolas Pröllochs, Kaoru Schwarzenegger, Philip
Pärnamets, Jay J. Van Bavel, and Stefan Feuerriegel. 2023. {``Negativity
Drives Online News Consumption.''} \emph{Nature Human Behaviour} 7:
812--22. \url{https://doi.org/10.1038/s41562-023-01538-4}.

\bibitem[\citeproctext]{ref-SimonsohnGruson_2024}
Simonsohn, Uri, and Hugo Gruson. 2024. \emph{Groundhog: Version-Control
for CRAN, GitHub, and GitLab Packages}.
\url{https://CRAN.R-project.org/package=groundhog}.

\bibitem[\citeproctext]{ref-TrisovicEtAl_2022}
Trisovic, Ana, Matthew K. Lau, Thomas Pasquier, and Mercè Crosas. 2022.
{``A Large-Scale Study on Research Code Quality and Execution.''}
Analysis. \emph{Scientific Data} 9 (60).
\url{https://doi.org/10.1038/s41597-022-01143-6}.

\bibitem[\citeproctext]{ref-renv}
Ushey, Kevin, and Hadley Wickham. 2025. \emph{Renv: Project
Environments}. \url{https://CRAN.R-project.org/package=renv}.

\bibitem[\citeproctext]{ref-Vilhuber_2021}
Vilhuber, Lars. 2021. {``Fireside Chat with AEA Data Editor:
Demystifying Reproducibility.''} Zenodo.
\url{https://doi.org/10.5281/zenodo.4450855}.

\bibitem[\citeproctext]{ref-VilhuberEtAl_2022}
Vilhuber, Lars, Marie Connolly, Miklós Koren, Joan Llull, and Peter
Morrow. 2022. {``A Template README for Social Science Replication
Packages.''} Technical note. Social Science Data Editors.
\url{https://doi.org/10.5281/zenodo.7293838}.

\end{CSLReferences}




\end{document}
